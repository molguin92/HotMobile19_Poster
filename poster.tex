%%%%%%%%%%%%%%%%%%%%%%%%%%%%%%%%%%%%
% template.tex
%
% Template for the KTHEEposter class.
%
% Original version: Mats Bengtsson, 28/5 2002
% Updated:
%   Mats Bengtsson April 2011 - July 2012
%   Manuel Olguín October 2018
%%%%%%%%%%%%%%%%%%%%%%%%%%%%%%%%%%%%

\documentclass[portrait, a1]{KTHEEposter}
\usepackage{lipsum}
\usepackage{listings}
\usepackage{enumitem}
\usepackage{caption}
\usepackage{indentfirst}
\lstset{basicstyle=\ttfamily, frame=single}

\kthlogo{kth_eng_cmyk}
\extralogo{img/cmu_csd_logo}
\kthcolor{KTHblue}

\begin{document}
    
    \title{\LARGE\bfseries Scaling on the Edge:\\A Benchmarking Suite For Human-in-the-Loop Applications}
    
    \author{\Large M. Olguín, J. Wang, M. Satyanarayanan, J. Gross}
    \maketitle
    
    \begin{pcolumns}[3]
        \begin{pcolumn}[2]
            \begin{pframe}
                Benchmarking human-in-the-loop application is complex given their nature, which heavily depends on the actions taken by the \emph{human} user.
                This limits reproducibility as well as feasibility of performance evaluations.
                In this paper we propose a methodology and present a benchmarking suite we call EdgeDroid  that can address these challenges.
                Our core idea rests on recording traces of these applications which are played out in a controlled fashion based on an underlying model of human behavior.
                The traces are then exposed to the original backend compute process of the respective human-in-the-loop application, generating realistic feedback.
                This allows for an automated system which greatly simplifies benchmarking large scale scenarios.
                After presenting this methodology and implementation, we demonstrate the approach on an example cognitive assistance application.
                The results show the utility of EdgeDroid as a tool for system designers, application developers and researchers alike.
            \end{pframe}
        \end{pcolumn}
    \end{pcolumns}
    
\end{document}